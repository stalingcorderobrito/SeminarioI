\documentclass[12pt,a4paper]{article}

% Paquetes básicos
\usepackage[spanish]{babel}
\usepackage[utf8]{inputenc}
\usepackage[T1]{fontenc}
\usepackage{lmodern}
\usepackage{setspace}             % Para interlineado
\usepackage{geometry}             % Para márgenes
\usepackage{fancyhdr}             % Encabezados y pies de página
\usepackage{graphicx}             % Inclusión de gráficos
\usepackage{array}                % Mejoras en tablas
\usepackage{booktabs}             % Tablas profesionales
\usepackage{longtable}            % Tablas que se extienden en varias páginas
\usepackage{hyperref}             % Hipervínculos
\hypersetup{
    colorlinks=true,
    linkcolor=black,
    urlcolor=blue,
    citecolor=blue,
    pdfborder={0 0 0}
}
\usepackage{tikz}
\usetikzlibrary{calc}
\usepackage{eso-pic}              % Para agregar elementos a todas las páginas
\usepackage{everypage}            % Para ejecutar código en cada página

% Configuración de márgenes
\geometry{
  top=2.5cm,
  bottom=2.5cm,
  left=3cm,
  right=2.5cm,
}

% Hook para agregar líneas verticales en cada página
\AddEverypageHook{%
  \begin{tikzpicture}[remember picture, overlay]
    % Línea azul a 1 cm del borde izquierdo
    \draw[blue, line width=2pt] 
      ($(current page.north west)+(1cm,0)$) -- 
      ($(current page.south west)+(1cm,0)$);
    % Línea roja a 1 cm del borde derecho
    \draw[red, line width=2pt] 
      ($(current page.north east)+(-1cm,0)$) -- 
      ($(current page.south east)+(-1cm,0)$);
  \end{tikzpicture}%
}

% Interlineado de 1.5
\onehalfspacing

% Encabezado y pie de página (se activarán en el documento, pero no en la portada)
\pagestyle{fancy}
\fancyhf{}
\fancyhead[R]{\thepage}
\fancyhead[L]{Seminario de Investigación}

% Título del documento (información para la portada)
\title{\textbf{Plantilla para Anteproyecto de Investigación}}
\author{Nombre del Investigador \\ Facultad/Institución}
\date{\today}

\begin{document}

%--------------------------------------------------
% Página de Portada Profesional
%--------------------------------------------------
\begin{titlepage}
  \thispagestyle{empty}  % Quitar encabezados y pies
  \begin{center}
    \vspace*{1cm}
    % Logo (ajusta la ruta y dimensiones)
    \includegraphics[width=0.2\textwidth]{logo.png}\\[1cm]
    
    % Nombre de la institución o universidad
    {\Large \textbf{Nombre de la Universidad o Institución}}\\[0.5cm]
    
    % Título del documento o proyecto
    {\Huge \bfseries \thetitle}\\[1.5cm]
    
    % Información del autor
    {\Large \textbf{\theauthor}}\\[0.5cm]
    
    % Datos adicionales (opcional)
    {\large Departamento o Carrera}\\[0.5cm]
    
    % Fecha
    {\large \thedate}\\[2cm]
    
    % Línea decorativa (opcional)
    \begin{tikzpicture}
      \draw[blue, line width=1.5pt] (0,0) -- (6,0);
      \draw[red, line width=1.5pt] (0,-0.3) -- (6,-0.3);
    \end{tikzpicture}
    
    \vfill
    % Pie de página (opcional)
    {\small Este anteproyecto ha sido preparado para el Seminario de Investigación}\\
    {\small Facultad/Departamento de [Nombre del Departamento]}
  \end{center}
\end{titlepage}

%--------------------------------------------------
% Inicio del Documento Principal
%--------------------------------------------------
\maketitle
\thispagestyle{empty}  % La página con el índice puede tener estilo distinto si se prefiere

\tableofcontents
\newpage

%%%%%%%%%%%%%%%%%%%%%%%%%%%%%%%%%%%%%%%%%%%%%%
% 1. Resumen Ejecutivo
%%%%%%%%%%%%%%%%%%%%%%%%%%%%%%%%%%%%%%%%%%%%%%
\section*{Resumen Ejecutivo}
\addcontentsline{toc}{section}{Resumen Ejecutivo}
\noindent
\textit{Escriba aquí el resumen ejecutivo.}

%%%%%%%%%%%%%%%%%%%%%%%%%%%%%%%%%%%%%%%%%%%%%%
% 2. Título
%%%%%%%%%%%%%%%%%%%%%%%%%%%%%%%%%%%%%%%%%%%%%%
\section{Título}
\noindent
\textbf{Título:} \\
\textit{Escriba el título de su investigación.}

%%%%%%%%%%%%%%%%%%%%%%%%%%%%%%%%%%%%%%%%%%%%%%
% 3. Problema de Investigación
%%%%%%%%%%%%%%%%%%%%%%%%%%%%%%%%%%%%%%%%%%%%%%
\section{Problema de Investigación}
\noindent
\textbf{Planteamiento:} \\
\textit{Describa detalladamente el problema a investigar.}

%%%%%%%%%%%%%%%%%%%%%%%%%%%%%%%%%%%%%%%%%%%%%%
% 4. Justificación y Delimitación de la Investigación
%%%%%%%%%%%%%%%%%%%%%%%%%%%%%%%%%%%%%%%%%%%%%%
\section{Justificación y Delimitación de la Investigación}
\noindent
\textbf{Justificación:} \\
\textit{Explique el porqué de la investigación, sustentando con fuentes y datos.} \\[1ex]
\textbf{Delimitación:} \\
\textit{Indique el alcance y los límites del estudio.}

%%%%%%%%%%%%%%%%%%%%%%%%%%%%%%%%%%%%%%%%%%%%%%
% 5. Beneficios para el País y Beneficiarios
%%%%%%%%%%%%%%%%%%%%%%%%%%%%%%%%%%%%%%%%%%%%%%
\section{Beneficios para el País y Beneficiarios}
\noindent
\textit{Describa los beneficios esperados y a quiénes están dirigidos.}

%%%%%%%%%%%%%%%%%%%%%%%%%%%%%%%%%%%%%%%%%%%%%%
% 6. Impacto Esperado
%%%%%%%%%%%%%%%%%%%%%%%%%%%%%%%%%%%%%%%%%%%%%%
\section{Impacto Esperado}
\noindent
\textit{Escriba el impacto que se espera lograr con la investigación.}

%%%%%%%%%%%%%%%%%%%%%%%%%%%%%%%%%%%%%%%%%%%%%%
% 7. Objetivos de la Investigación
%%%%%%%%%%%%%%%%%%%%%%%%%%%%%%%%%%%%%%%%%%%%%%
\section{Objetivos de la Investigación}
\subsection*{Objetivo General}
\addcontentsline{toc}{subsection}{Objetivo General}
\noindent
\textit{Defina el propósito global de la investigación.}

\subsection*{Objetivos Específicos}
\addcontentsline{toc}{subsection}{Objetivos Específicos}
\begin{itemize}
    \item \textit{Objetivo 1: Describir, analizar, evaluar, etc. (según corresponda).}
    \item \textit{Objetivo 2: Describir, analizar, evaluar, etc.}
    \item \textit{Objetivo 3: Describir, analizar, evaluar, etc.}
\end{itemize}

%%%%%%%%%%%%%%%%%%%%%%%%%%%%%%%%%%%%%%%%%%%%%%
% 8. Marco Teórico
%%%%%%%%%%%%%%%%%%%%%%%%%%%%%%%%%%%%%%%%%%%%%%
\section{Marco Teórico}
\noindent
\textit{Describa el estado del arte y la fundamentación teórica que sustenta la investigación.}

%%%%%%%%%%%%%%%%%%%%%%%%%%%%%%%%%%%%%%%%%%%%%%
% 9. Método
%%%%%%%%%%%%%%%%%%%%%%%%%%%%%%%%%%%%%%%%%%%%%%
\section{Método}
\noindent
\textbf{Materiales:} \textit{Especifique los materiales experimentales.} \\[1ex]
\textbf{Métodos y Técnicas:} \textit{Describa el método a emplear en cada fase del proyecto.} \\[1ex]
\textbf{Procedimientos y Actividades:} \textit{Explique detalladamente cómo, dónde y cuándo se realizarán las actividades.} \\[1ex]
\textbf{Aspectos Éticos:} \textit{Incorpore la normativa y consideraciones éticas pertinentes.}

\noindent
\textbf{Línea de Investigación:} \\
\textit{Indique la línea y área de estudio, vinculada a las líneas aprobadas.}

%%%%%%%%%%%%%%%%%%%%%%%%%%%%%%%%%%%%%%%%%%%%%%
% 10. Productos o Resultados
%%%%%%%%%%%%%%%%%%%%%%%%%%%%%%%%%%%%%%%%%%%%%%
\section{Productos o Resultados}
\noindent
\begin{enumerate}
    \item \textit{Producto 1}
    \item \textit{Producto 2}
    \item \textit{Producto 3}
\end{enumerate}

%%%%%%%%%%%%%%%%%%%%%%%%%%%%%%%%%%%%%%%%%%%%%%
% 11. Estrategia de Divulgación
%%%%%%%%%%%%%%%%%%%%%%%%%%%%%%%%%%%%%%%%%%%%%%
\section{Estrategia de Divulgación de los Resultados}
\noindent
\textit{Explique cómo se comunicarán los resultados, tanto al ámbito científico como al público en general.}

%%%%%%%%%%%%%%%%%%%%%%%%%%%%%%%%%%%%%%%%%%%%%%
% 12. Cronograma de Actividades
%%%%%%%%%%%%%%%%%%%%%%%%%%%%%%%%%%%%%%%%%%%%%%
\section{Cronograma de Actividades}
\begin{center}
\begin{tabular}{|c|c|c|c|c|c|c|c|c|c|c|c|c|}
\hline
\textbf{Actividad} & 1 & 2 & 3 & 4 & 5 & 6 & 7 & 8 & 9 & 10 & 11 & 12 \\
\hline
\textbf{Fase I} & X & X &  &  &  &  &  &  &  &  &  &  \\
\hline
\textbf{Fase II} &  &  & X & X &  &  &  &  &  &  &  &  \\
\hline
\end{tabular}
\end{center}

%%%%%%%%%%%%%%%%%%%%%%%%%%%%%%%%%%%%%%%%%%%%%%
% 13. Presupuesto (en Balboas)
%%%%%%%%%%%%%%%%%%%%%%%%%%%%%%%%%%%%%%%%%%%%%%
\section{Presupuesto (en Balboas)}
\begin{center}
\begin{tabular}{|l|c|c|c|c|}
\hline
\textbf{Etapa del Proyecto} & \textbf{Recursos (Equipos/Materiales)} & \textbf{Recursos Humanos} & \textbf{Salidas de Campo} & \textbf{Total} \\
\hline
I & B/ \underline{\hspace{1.5cm}} & B/ \underline{\hspace{1.5cm}} & B/ \underline{\hspace{1.5cm}} & B/ \underline{\hspace{1.5cm}} \\
\hline
II & B/ \underline{\hspace{1.5cm}} & B/ \underline{\hspace{1.5cm}} & B/ \underline{\hspace{1.5cm}} & B/ \underline{\hspace{1.5cm}} \\
\hline
\end{tabular}
\end{center}

%%%%%%%%%%%%%%%%%%%%%%%%%%%%%%%%%%%%%%%%%%%%%%
% 14. Formato para el Documento en General
%%%%%%%%%%%%%%%%%%%%%%%%%%%%%%%%%%%%%%%%%%%%%%
\section{Formato para el Documento en General}
\begin{itemize}
    \item \textbf{Márgenes:} Superior, inferior y derecha: 2.5 cm; izquierda: 3 cm.
    \item \textbf{Fuente:} Arial, 12 pt.
    \item \textbf{Espaciado:} 1.5 (espacio y medio).
    \item \textbf{Paginación:} Números arábigos en la esquina superior derecha.
    \item \textbf{Citas:} Utilizar un estilo de citación único (ej. APA, IEEE, etc.) y que todas las fuentes citadas en el texto aparezcan en la bibliografía.
    \item \textbf{Cuadros, Tablas y Gráficos:} Cada uno debe tener numeración, título (en cursiva para el título, no para la palabra “Cuadro”, “Tabla” o “Gráfico”) y la fuente.
\end{itemize}

%%%%%%%%%%%%%%%%%%%%%%%%%%%%%%%%%%%%%%%%%%%%%%
% 15. Bibliografía
%%%%%%%%%%%%%%%%%%%%%%%%%%%%%%%%%%%%%%%%%%%%%%
\section{Bibliografía}
\bibliographystyle{apalike}
\bibliography{referencias}

%%%%%%%%%%%%%%%%%%%%%%%%%%%%%%%%%%%%%%%%%%%%%%
% 16. Anexos
%%%%%%%%%%%%%%%%%%%%%%%%%%%%%%%%%%%%%%%%%%%%%%
\section{Anexos}
\noindent
\textit{Adjunte en esta sección todos los documentos complementarios requeridos.}

\end{document}
