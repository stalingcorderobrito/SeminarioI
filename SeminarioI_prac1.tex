\documentclass[12pt]{article}
\usepackage[utf8]{inputenc}
\usepackage[spanish]{babel}
\usepackage{geometry}
\geometry{a4paper, margin=1in}
\usepackage{array}
\usepackage{booktabs}
\usepackage{multicol}
\usepackage{amssymb} % Para usar \checkmark
\usepackage[T1]{fontenc} % Para mejorar la visualización de caracteres acentuados
\usepackage{lmodern}   
\usepackage{wasysym} % Para el emoji \smiley  % Fuente vectorial que soporta Unicode

\title{\textbf{Seminario de Investigación I \\ Operacionalización de Variables}}

\author{\textbf{Docente:} Staling Cordero Brito \\
\textbf{Valor de la práctica:} 10 puntos}
\date{\textbf{Fecha de entrega:} 4-abril-2025}

\begin{document}

\maketitle

\section*{Objetivo}
El objetivo es aprender a operacionalizar variables: identificar conceptos clave, definir variables y desarrollar indicadores y escalas de medición a partir de problemas de investigación. La presentación se realizará en la siguiente clase, dedicando —como máximo— 15 minutos por grupo.

\section*{Instrucciones para la tarea}

\checkmark \textbf{Formación de equipos:} Se han conformado 8 grupos (los mismos que tienen en la clase). \\

\checkmark \textbf{Asignación de temas:} A cada grupo se le ha asignado un problema de investigación distinto (ver los temas asignados). 

\begin{enumerate}
    
 \item \textbf{Desarrollo de la actividad:}
    \begin{enumerate}
        \item Identifiquen las variables principales y secundarias del problema asignado (por ejemplo, variables independiente y dependiente). 
        \item Redacten una definición conceptual breve para cada variable.
        \item Desglosen cada variable en sus dimensiones correspondientes.
        \item Propongan indicadores específicos para cada dimensión.
        \item Determinen la escala de medición adecuada (nominal, ordinal, intervalo o razón, etc.) para cada indicador.
        \item Elaboren un cuadro resumen que incluya: Variables, definición conceptual, dimensiones, indicadores y escala de medición.
    \end{enumerate}
    
    \item \textbf{Entrega y presentación:}
    \begin{itemize}
        \item La tarea se realizará en grupo y se debe entregar en formato digital, via Classroom, el 4 de abril de los corrientes (como máximo a las 5:00 pm).
        \item Cada grupo dispondrá de 15 minutos (o menos) para exponer su cuadro resumen y explicar el proceso de operacionalización.
    \end{itemize}
    
    \item \textbf{Retroalimentación:} Durante las exposiciones, cada grupo, realizará una discusión grupal para resaltar los enfoques y buenas prácticas observadas.
\end{enumerate}

\section*{Temas asignados a los grupos}
\begin{enumerate}
    \item \textbf{Grupo 1:} ¿Cómo influye el uso de tecnología educativa en el rendimiento académico de estudiantes de preuniversitario?
    \item \textbf{Grupo 2:} ¿De qué manera la participación en actividades extracurriculares afecta la motivación escolar?
    \item \textbf{Grupo 3:} ¿Cómo impacta la implementación de metodologías activas en el aprendizaje?
    \item \textbf{Grupo 4:} ¿Qué efecto tiene la capacitación docente en el uso de TIC en la calidad educativa?
    \item \textbf{Grupo 5:} ¿Cómo influye el ambiente familiar en la orientación vocacional de estudiantes de preuniversitario?
    \item \textbf{Grupo 6:} ¿Cuál es la relación entre el clima escolar y el desempeño académico?
    \item \textbf{Grupo 7:} ¿Cómo afecta el uso de plataformas digitales en la gestión del aprendizaje en aulas de preuniversitario?
    \item \textbf{Grupo 8:} ¿De qué manera la integración de habilidades socioemocionales en el currículo mejora la convivencia escolar?
\end{enumerate}

\section*{Nota}
\begin{itemize}
    \item La práctica se evaluará en función de la claridad y profundidad del análisis epxresado, así como de la calidad de la presentación.
    \item Se estará compartiendo por Clasroom un material de apoyo sobre cómo operacionalizar variables.
    
\end{itemize}

\begin{center}
    \Huge \textbf{¡A inspirar y transformar! \smiley}
\end{center}


\end{document}
